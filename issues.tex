\section*{Practical Issues}
%\markboth{Practical Issues}{Practical Issues}
%\addcontentsline{toc}{section}{Practical Issues}

\subsection*{Last Update: \today}


\subsection*{Contacts}
\textbf{Email}: \url{colin-young@live.com}

The latest update can be found via: \url{https://github.com/yuhao-yang-cy/asphysics}

\subsection*{About the Notes}

This is a set of very concise lecture notes written for CIE AS-level Physics (syllabus code 9702). Presumably the target audience of the notes are students studying the relevant course.

These notes are supposed to be self-contained. I believe I have done my best to make the lecture notes reflect the spirit of the syllabus set by the Cambridge International Examination Board. Apart from the essential derivations and explanations, I also included a handful of worked examples and problem sets, so that you might get some rough idea about the styles of questions that you might encounter in the exams. If you are a student studying this course, I believe these notes could help you get well-prepared for the exams.

Despite the fact that the notes are supposed to serve the CIE candidates, the world of physical science is so rich and wonderful, and the lots of fascinating details of the nature and the universe produce insights and understandings of many things all around us. I would love to share and selectively include in the notes a small part that I know, although some of the materials are beyond the CIE requirements. I would apologize ahead as in some chapters I just could not help myself filling in materials that I find interesting. There are pages where footnotes take more space than the main body.

Many materials in the notes are borrowed from the textbooks listed on the next page and the know-it-all internet. I strongly recommend those readers who want to know more to take a close look at the list of references.

Throughout these notes, key concepts are marked red, key definitions and important formulas are boxed. The comments I would like to make on specific topics are followed by a left hand symbol (\ding{43}). Any non-examinable material that usually show up in the notes are pointed out explicitly. Anything labelled with a star sign is a warning sign that it is beyond the CIE syllabus.

I don't think anyone can learn physics without doing sufficient exercises, that is why I have offered a few worked examples. There are also two versions of the notes, one with an end-of-chapter question section for each chapter and one without. You can freely choose which version to satisfy your personal need, but I would recommend you to get your hands on as many problems as you can to test your understanding.

These lecture notes are not yet complete. I will follow them up with the development of the course.

Also very importantly, I am certain that there are countless typos in the notes. If you spot any errors, please let me know.



\subsection*{Literature}

I borrow heavily from the following resources:

\begin{itemize}
	\item[-] Cambridge International AS and A Level Physics Coursebook, by \textit{David Sang, Graham Jones, Richard Woodside} and \textit{Gurinder Chadha}, Cambridge University Press
	
	\item[-] International A Level Physics Revision Guide, by \textit{Richard Woodside}, Hodder Education
	
	\item[-] Longman Advanced Level Physics, by \textit{Kwok Wai Loo},	Pearson Education South Asia
	
	\item[-] Conceptual Physics (10$^\text{th}$ Edition), by \textit{Paul G. Hewitt}, Pearson International Education
	
	\item[-] Fundamentals of Physics, by \textit{Robert Resnick, David Halliday} and \textit{Kenneth S. Krane}, John Wiley \& Sons
	
	\item[-] Past Papers of Cambridge Internation A-Level Physics Examinations
	
	\item[-] HyperPhysics Website: \url{http://hyperphysics.phy-astr.gsu.edu/hbase/index.html}
	
	\item[-] Wikipedia Website: \url{https://en.wikipedia.org}
\end{itemize}

\subsection*{Copyright}

This work is offered under a \textbf{CC BY-NC} (Creative Commons Attribution-Non-Commercial) license. You may remix, adapt, and build upon this work, as long as the attribution is given and the new work is non-commercial.


%\subsection*{Recommended Reading}
