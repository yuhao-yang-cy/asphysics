\section*{Practical Issues}
%\markboth{Practical Issues}{Practical Issues}
%\addcontentsline{toc}{section}{Practical Issues}

\subsection*{Last Update: \today}


\subsection*{Contacts}
\textbf{Email}: \url{colin-young@live.com}

The latest update can be found via: \url{https://github.com/yuhao-yang-cy/asphysics}

\subsection*{About the Notes}

The contents of the notes are consistent with the CIE A-Level physics syllabus. I attempt to systematically cover all the key points in the syllabus with brief but sufficient explanations. The notes should be able to serve as a self-contained study guide for the AS CIE course.

I am still working on the notes. The latest release is far from complete as it only covers the first few chapters. I hope I will follow up the other chapters before the end of this year.

If you spot any errors, please let me know.

%Despite the fact that the notes are supposed to serve the CIE candidates, the world of physical science is so rich and wonderful, and the lots of fascinating details of the nature and the universe produce insights and understandings of many things all around us, I would be love to share and selectively include in the notes a small part that I know, although some of the materials are beyond the CIE requirements.
%
%Many materials in the notes are borrowed from the textbooks listed on the next page and the know-it-all internet. I strongly recommend those readers who would like to know more to take a close look at the list of references.
%
%Throughout these notes, key concepts are marked red, important formulas are boxed. The comments I would like to make on specific topics are followed by a left hand symbol (\ding{43}). Materials beyond the CIE syllabus are labelled with an asterisk (\ding{86}), which usually show up in the footnotes.
%
%I don't think anyone can learn physics without doing sufficient exercises, that is why I have offered a few worked examples and some relevant problems. I have adopted a `Michelin Guide' style rating scheme to the problems, with numbers of stars to suggest the level of difficulty. A problem without a star is an essential exercise that every reader should practice. One-star problems are more difficult and require a clear understanding of several concepts. Two-star problems could be very challenging and require a much deeper understanding together with some natural intuitions. They are not likely to show up in the real CIE exam, but I believe the hardcore problems could give insights for further studies in physics.
%
%These lecture notes are not yet complete. I will follow them up with the development of the course.
%
%Also very importantly, I am certain that there are countless typos in the notes. If you spot any errors, please let me know.
\subsection*{Literature}

I borrow heavily from the following resources:

\begin{itemize}
	\item[-] Cambridge International AS and A Level Physics Coursebook, by \textit{David Sang, Graham Jones, Richard Woodside} and \textit{Gurinder Chadha}, Cambridge University Press
	
	\item[-] International A Level Physics Revision Guide, by \textit{Richard Woodside}, Hodder Education
	
	\item[-] Longman Advanced Level Physics, by \textit{Kwok Wai Loo},	Pearson Education South Asia
	
	\item Conceptual Physics (10$^\text{th}$ Edition), by \textit{Paul G. Hewitt}, Pearson International Education
	
	\item Physics (5$^\text{th}$ Edition), by \textit{Robert Resnick, David Halliday} and \textit{Kenneth S. Krane}, John Wiley \& Sons 2002
	
	\item[-] Past Papers of Cambridge Internation A-Level Physics Examinations
	
	\item[-] HyperPhysics Website: \url{http://hyperphysics.phy-astr.gsu.edu/hbase/index.html}
	
	\item[-] Wikipedia Website: \url{https://en.wikipedia.org}
\end{itemize}

\subsection*{Copyright}

This work is offered under a \textbf{CC BY-NC} (Creative Commons Attribution-Non-Commercial) license. You may remix, adapt, and build upon this work, as long as the attribution is given and the new work is non-commercial.


\subsection*{Recommended Reading}
