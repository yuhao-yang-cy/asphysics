\section*{Appendix A. Glossary}
\addcontentsline{toc}{part}{Appendix A. Glossary}
\rhead{ \fancyplain{}{GLOSSARY} }

\vspace*{-2em}
{
	\small
	\setdefaultleftmargin{8pt}{4pt}{}{}{}{}

\begin{enumerate}

\item {\large measurements}

\begin{compactitem}

\item \emph{SI base units}: second (time), metre (length), kilogram (mass), kelvin (temperature), ampere (electric current), mole (amount of substance)

\item \emph{systematic error}: all readings are greater or smaller than true value by the same amount

\item \emph{random error}: readings fluctuate above or below some average value, data points are scattered around best-fit curve

\item \emph{accuracy}: whether the measurement is close to the true value

\item \emph{precision}: whether repeated readings are close to one another

\item \emph{how to reduce systematic error}: check for zero error/calibrate equipment before measurements

\item \emph{how to reduce random error}: repeat readings and take average

\end{compactitem}

% % % % % % % % % % % % % % % % % % % % % % %

\item {\large kinematics}

\begin{compactitem}

\item \emph{distance}: total length of actual path taken by a moving object (scalar)

\item \emph{displacement}: straight-line distance in a specific direction (vector) 

\item \emph{velocity}: rate of change in displacement (vector) ($v=\frac{\Delta s}{\Delta t}$)

\item \emph{acceleration}: rate of change in velocity (vector) ($a=\frac{\Delta v}{\Delta t}$)

\item \emph{relations between motion graphs}: gradient of $s$-$t$ graph gives velocity; gradient of $v$-$t$ graph gives acceleration; area under $v$-$t$ graph gives change in displacement


\end{compactitem}


% % % % % % % % % % % % % % % % % % % % % % %

\item {\large force \& motion}

\begin{compactitem}
	
\item \emph{resultant/net force}: a single force that has the same effect as all forces combined
	
\item \emph{Newton's first law (law of inertia)}: an object stays at rest or moves with constant velocity if there is no resultant force
	
\item \emph{Newton's second law}: acceleration of an object is proportional to resultant force, and inversely proportional to its mass ($F_\text{net}=ma$)

\item \emph{Newton's third law/action-reaction principle}: action and reaction are of the same type, act with equal magnitude in opposite directions

\item \emph{equilibrium of forces}: resultant force is zero (in case of two forces acting , they are equal but opposite; in case of three forces, they form a closed vector triangle)

\item \emph{weight}: a force of gravity due to gravitational attraction by the earth, always acting downwards

\item \emph{distinguish between mass and weight}: mass is an intrinsic property of object which does not change with location, has no direction; weight is a force of gravity which depends on the gravitational field, acts towards centre of planet

\item \emph{acceleration of free fall}: in absence of air, acceleration of falling objects/projectiles is constant and always points vertically downwards

\item \emph{free fall in air}: as speed of falling object increases, air resistance increases, so net force decreases, so acceleration gradually decreases to zero, speed tends to constant

\item \emph{terminal speed}: when air resistance is in equilibrium with weight of falling object, it falls with constant speed called the terminal speed

\item \emph{motion of object sliding down an inclined slope}: acceleration produced by component of weight along the slope (subtract friction if surface is not smooth) ($mg\sin\theta - f = ma$)

\item \emph{torque/moment of a force}: force $\times$ perpendicular distance from pivot to line of force

\item \emph{torque of a couple}: one force of the pair $\times$ perpendicular distance between the two forces

\item \emph{centre of gravity/mass}: point at which entire weight of an object can be considered to act

\item \emph{conditions for mechanical equilibrium}: no resultant force, and no resultant torque/moment

\item \emph{principle of moments}: sum of all clockwise moments is equal to sum of all anticlockwise moments about any point
	
\end{compactitem}


% % % % % % % % % % % % % % % % % % % % % % %

\item {\large work \& energy}

\begin{compactitem}
	
\item \emph{work done}: force $\times$ displacement moved in direction of the force (scalar) ($W=Fs$)
	
\item \emph{power}: work done per unit time (scalar) ($P=\frac{\Delta W}{\Delta t}$)

\item \emph{how to find work done by varying force}: find area under force-displacement graph
	
\item \emph{gravitational potential energy}: energy due to an object's position in a gravitational field

\item \emph{elastic potential energy}: energy due to change of shape/deformation for an object

\item \emph{kinetic energy}: energy possessed by an object due to motion

\item \emph{conservation of energy}: energy cannot be created or destroyed, it can only transform from one form into another, total amount of energy is constant for any closed system

\item \emph{energy changes for object sliding down a slope}: loss in potential energy equals gain in kinetic energy plus work down against friction

\item \emph{efficiency}: ratio of useful power/energy to total power/energy input
	
\end{compactitem}


% % % % % % % % % % % % % % % % % % % % % % %
\newpage

\item {\large momentum}

\begin{compactitem}
	
\item \emph{momentum}: mass $\times$ velocity (vector) ($p=mv$)

\item \emph{force}: rate of change in momentum ($F=\frac{\Delta p}{\Delta t}$)

\item \emph{Newton's second law (alternative description)}: resultant force acting on an object equals rate of change in its momentum
	
\item \emph{conservation of momentum}: if there is no net external force (isolated system), total momentum of a system of objects remains constant
	
\item \emph{elastic collision}: no kinetic energy loss during the collision

\item \emph{inelastic collision}: kinetic energy is lost as objects undergo deformation during collision

\item \emph{how to identify an elastic collision}: relative speed of approach before collision is equal to relative speed of separation after collision
	
\end{compactitem}


% % % % % % % % % % % % % % % % % % % % % % %

\item {\large solids \& fluids}

\begin{compactitem}
	
\item \emph{elasticity}: when external force is removed, object can return to its original shape/length
	
\item \emph{elastic limit}: extension for an object beyond which it cannot restore to original shape
	
\item \emph{Hooke's law}: extension/compression of an ideal spring is proportional to force applied (within limit of proportionality) ($F=kx$)

\item \emph{stress}: tensile force per unit cross-sectional area ($\sigma = \frac{F}{A}$)

\item \emph{strain}: extension per unit original length ($\epsilon = \frac{x}{L}$)

\item \emph{Young's modulus}: ratio of stress to strain ($E=\frac{\sigma}{\epsilon}=\frac{FL}{Ax}$)

\item \emph{heating effect when stretching and unloading rubber bands}: more work is done to stretch rubber band than that is given out as rubber band unloads, the difference becomes thermal energy
	
\item \emph{pressure}: force per unit area ($p=\frac{F}{A}$)
	
\item \emph{density}: mass per unit volume ($\rho = \frac{m}{V}$)
	
\item \emph{pressure in fluids}: self-weight of fluid produces pressure that increases linearly with depth ($p=\rho g h$, if density $\rho$ is constant)

\item \emph{cause of upthrust}: pressure difference between top and bottom surface for immersed object
	
\end{compactitem}

% % % % % % % % % % % % % % % % % % % % % % %

\item {\large waves}

\begin{compactitem}
	
\item \emph{displacement}: distance from equilibrium position for a particle
	
\item \emph{amplitude}: maximum displacement throughout the oscillation
	
\item \emph{period}: time taken for one complete oscillation

\item \emph{frequency}: number of oscillations per unit time

\item \emph{wavelength}: distance between two neighbouring peaks/crests

\item \emph{mechanical waves}: waves that cannot travel without a medium

\item \emph{electromagnetic waves}: wave that involves vibration of electric and magnetic fields

\item \emph{properties of electromagnetic waves}: can travel through vacuum/free space; constant speed in vacuum $c=3.0\times10^8 \text{m/s}$; cannot be deflected by electric or magnetic fields, etc.

\item \emph{electromagnetic spectrum}: $\gamma$-ray, X-ray, ultraviolet, visible light, infra-red, microwave, radio wave (in order of increasing  wavelength, or decreasing frequency)

\item \emph{longitudinal waves}: direction of vibration/displacement of particles is parallel to direction of propagation of energy/wave motion

\item \emph{examples of longitudinal waves}: sound waves

\item \emph{transverse waves}: direction of vibration/displacement of particles is perpendicular to direction of propagation of energy/wave motion

\item \emph{examples of transverse waves}: electromagnetic waves, surface wave on water, wave travelling on a string, etc.

\item \emph{intensity}: power transferred per unit area

\item \emph{factors that determine intensity}: intensity is proportional to amplitude squared ($I \propto A^2$); for a spherical wave, intensity is inversely proportional to distance from source squared ($I \propto \frac{1}{r^2}$)

\item \emph{Doppler effect}: relative (radial) motion between wave source and observer causes a change in the apparent/observed frequency

\item \emph{cause of Doppler effect}: when wave source moves closer to/away from observer, apparent wavelength becomes shorter/longer, so observed frequency becomes higher/lower
	
\end{compactitem}

% % % % % % % % % % % % % % % % % % % % % % %

\item {\large superposition of waves}

\begin{compactitem}
	
\item \emph{principle of wave superposition}: when two waves meet/overlap, resultant displacement is the sum of displacements of each individual
	
\item \emph{condition for constructive superposition}: peak meets peak; phase difference between two waves $\Delta \phi = 0, 2\pi, 4\pi, \dots$; path difference $\Delta L = 0, \lambda, 2\lambda, \dots$

\item \emph{condition for destructive superposition}: peak meets trough; phase difference between two waves $\Delta \phi = \pi, 3\pi, 5\pi, \dots$; path difference $\Delta L = \frac{1}{2}\lambda, \frac{3}{2}\lambda, \frac{5}{2}\lambda,  \dots$

\item \emph{monochromatic light}: light of single wavelength/frequency

\item \emph{diffraction}: as wave passes through a narrow slit/gap/obstacle, it bends and spreads out

\item \emph{increase effect of diffraction}: increase wavelength or reduce size of slit, greatest diffraction when wavelength is close to width of slit

\item \emph{coherence}: wave sources have constant phase difference (at any time)

\item \emph{how an interference pattern is formed}: when two coherent waves meet and superimpose, stable regions of constructive superposition are formed at points where $\Delta \phi = 0, 2\pi, 4\pi, \dots$ or $\Delta L = 0, \lambda, 2\lambda, \dots$, maxima are observed

\item \emph{condition for interference}: two waves overlap/meet together, waves are of same type, sources are coherent

\item \emph{double-slit experiment}: coherent light passes through two narrow slits and meet at screen, alternating bright and dark fringes are observed, fringe separation is given by $x=\frac{\lambda D}{d}$, where $d$ is separation between slits and $D$ is slit-to-screen distance (also see \emph{interference})

\item \emph{effect of increasing width of slit}: intensity of emergent light increases if slit is made wider, so bright fringes become brighter; if width of both slits are increased together, still have complete cancellation, so dark fringes remain dark

\item \emph{diffraction grating}: light incident on a system of many close slits to produce spectra by interference, angle $\theta$ of $n^\text{th}$ order maxima is given by $d\sin\theta = n\lambda$, where $d$ is separation between neighbouring slits

\item \emph{passing white light through diffraction grating}: all wavelengths have zero path difference at $\theta=0$, so $0^\text{th}$ order is white; to produce $1^\text{st}$ order, longer wavelength needs larger path difference, red/violet has maxima at greater/smaller angle, $1^\text{st}$ order is a continuous spectrum

\item \emph{how a stationary wave is formed}: two progressive waves travelling in opposite directions meet and superimpose, if they are of same frequency and wave speed, they form a stationary wave (stable regions of constructive and destructive interference)

\item \emph{node}: point that has zero displacement

\item \emph{anti-node}: point that vibrates with greatest amplitude

\item \emph{differences between progressive wave and stationary wave}: progressive wave can transfer energy, stationary wave cannot; amplitudes of different points in a progressive wave are the same, but phase difference can be different; amplitudes of different points in a stationary wave can be all different, but phase difference is either 0 or $\pi$

\item \emph{measuring wavelength using stationary waves}: send wave towards a reflector, adjust position of reflector until a point with zero intensity is found, then slowly move the detector until the next minima, distance $d$ moved out is the node-to-node distance, so wavelength $\lambda = 2d$
	
\end{compactitem}

% % % % % % % % % % % % % % % % % % % % % % %

% \item {\large electric fields}

% \begin{compactitem}
	
% \item \emph{electric field}: a region of space in which a charged object/particle experiences a force
	
% \item \emph{electric field strength}: electric force per unit positive charge ($E=\frac{F}{q}$)

% \item \emph{electric field lines}: a way to graphically show pattern of an electric field, field lines point in same direction as the field, spacing between lines gives information about magnitude of field strength
	
% \item \emph{direction of electric force on small charge}: for $+q$, same direction as field (line); for $-q$, opposite to field (line)
	
% \end{compactitem}

% % % % % % % % % % % % % % % % % % % % % % %

\item {\large current electricity}

\begin{compactitem}
	
\item \emph{electric current}: charge flow per unit time ($I=\frac{\Delta Q}{\Delta t}$)

\item \emph{electric charge}: current $\times$ time ($Q=It$)
	
\item \emph{potential difference (p.d.) between two points}: energy needed to move a unit positive charge from one point to another ($V=\frac{\Delta W}{\Delta q}$)
	
\item \emph{e.m.f. of a supply/battery}: energy gain by unit positive charge as it moves though the battery

\item \emph{distinguish between p.d. and e.m.f.}: p.d. across a resistor/component is amount of electrical energy converted into other forms (heat, light, mechanical, etc.) per unit charge, e.m.f. of a cell is amount of electrical energy converted from other forms (chemical, nuclear, mechanical, etc.) per unit charge

\item \emph{resistance}: ratio of potential difference to electric current ($R=\frac{V}{I}$)

\item \emph{factors that determine resistance}: resistance is proportional to length of conductor, inversely proportional to cross sectional area, and also depends on material ($R=\frac{\rho L}{A}$)

\item \emph{ohmic conductor}: resistance remains constant as p.d./current varies, $I$-$V$ characteristic graph is a straight light through origin

\item \emph{resistance of metallic conductors at larger currents}: increasing current produces more heating effect, higher temperature causes resistance of metal to increase

\item \emph{Kirchhoff's first law}: for any point in a circuit, sum of currents flowing into that point equals sum of currents flowing out

\item \emph{Kirchhoff's second law}: for any closed loop in a circuit, sum of all e.m.f.'s of supplies equals sum of all p.d. drops across components

\item \emph{Kirchhoff's laws and conserved quantities}: first law is related to charge conservation, second law is related to energy conservation

\item \emph{properties of components connected in series}: same current / p.d. is shared

\item \emph{properties of components connected in parallel}: same p.d. / current is shared

\item \emph{why terminal p.d. is less than e.m.f.}: as current flows through a cell, there exist lost volts in internal resistance, so terminal p.d. falls

\item \emph{internal resistance of a cell}: ratio of lost volts to current through cell

\item \emph{p.d. across a variable resistor in a potential divider circuit}: as variable resistor $R_v$ increases, current decreases, p.d. across fixed resistor $R_0$ increases, so p.d. across $R_v$ decreases
	
\end{compactitem}

% % % % % % % % % % % % % % % % % % % % % % %

\item {\large particle physics}

\begin{compactitem}
	
\item \emph{nucleus}: positively-charged core at centre of an atom, tiny but concentrates almost all the mass, made of nucleons

\item \emph{nucleon}: either a proton or a neutron

\item \emph{experimental results of $\alpha$-particle scattering experiment}: most $\alpha$-particles pass through gold foil with almost no deflection, but very few undergo large-angle deflections

\item \emph{atomic structure}: most space is empty, almost entire mass is concentrated in the nucleus

\item \emph{nuclide notation}: $^A_Z X$, $Z$ denotes charge/proton number, $A$ denotes mass/nucleon number

\item \emph{proton number/charge number $Z$}: number of protons, or number of electric charges, charge of particle is $+Ze$, where $e$ is elementary charge

\item \emph{nucleon number/mass number $A$}: total number of protons and neutrons, mass of nucleus/atom is $Au$, where $u$ is unified atomic mass

\item \emph{isotope}: nuclei/atoms with same number of protons but different number of neutrons

\item \emph{radioactive decay}: random emission of $\alpha$-, $\beta$- or/and $\gamma$-radiation from unstable nuclei

%\item \emph{randomness}: decay of a nucleus is not predictable/uncertain, it can only be described by probability

%\item \emph{spontaneity}: rate of decay is independent of external conditions such as temperature, pressure, electromagnetic field, neighbouring nuclei

\item \emph{ionising power}: radioactive radiation knock out electrons from air molecules, causing air molecules to become ionised

\item \emph{properties of $\alpha$-particle}: helium nucleus (2 protons and 2 neutrons), charge $+2e$, mass $4u$, can travel through only a few cm of air, stopped by thin paper, very strong ionising ability

\item \emph{properties of $\beta$-particle}: fast-moving electrons, charge $-e$, very small mass, stopped by a few mm of aluminium, strong ionising ability

\item \emph{properties of $\gamma$-radiation}: high-frequency electromagnetic wave, no charge, no mass, partially stopped by a few cm of lead/concrete, very penetrating, but weak ionising ability

\item \emph{$\alpha$-decay}: $^A_Z X \longrightarrow ^{A-4}_{Z-2}Y + ^4_2 \alpha$

\item \emph{$\beta$-decay}: $^A_Z X \longrightarrow ^{\phantom{0+}A}_{Z+1}Y + ^{\phantom{-}0}_{-1} \beta + ^0_0\bar{\nu}$, or $^1_0 \text{n} \longrightarrow ^1_1 \text{p} + ^{\phantom{-}0}_{-1} e + ^0_0\bar{\nu}$

\item \emph{why $\beta$-particles have a range of speeds}: anti-neutrinos also carry off energies released from decay process, so $\beta$-particles have a range of kinetic energies

\item \emph{conserved quantities in nuclear reactions}: electric charge, mass-energy, momentum, charge number, mass number

\item \emph{why mass is not conserved in nuclear reactions}: mass-energy is conserved, reduction in total mass becomes kinetic energy of product particles or electromagnetic energy of $\gamma$-radiation

\item \emph{hadron}: a family of particles affected by strong nuclear force

\item \emph{examples of hadrons}: proton, neutron, etc.%, $\pi$, $K$, $\Sigma$, $\Lambda$, etc.

\item \emph{lepton}: a family of particles not affected by strong nuclear force

\item \emph{examples of leptons}: electron, neutrino%, muon, etc.

\item \emph{properties of an anti-particle}: equal mass, but opposite charge%; when a particle and its anti-particle meet, they annihilate each other and produce $\gamma$-radiation

\item \emph{quark}: fundamental particles that are affected by strong nuclear force, they come in different flavours: up (u), down (d), strange (s), charm(c), top(t), bottom(b)

\item \emph{charge of quark}: up, charm, top $\left( +\frac{2}{3}e \right)$; down strange, bottom $\left( -\frac{1}{3}e \right)$

\item \emph{hadron}: composite particle formed by quarks, affected by strong nuclear force

\item \emph{baryon}: a type of hadron made up of three quarks (qqq)

\item \emph{meson}: a type of hadron made up of a quark and an anti-quark (q$\bar{\text{q}}$)

\item \emph{structure of proton and neutron}: proton (uud), neutron (udd)

\item \emph{$\beta^-$-decay}: $^1_0 \text{n} \longrightarrow ^1_1 \text{p} + ^{\phantom{-}0}_{-1} e^- + ^0_0\bar{\nu}$, or $\text{d} \longrightarrow \text{u} + e^- + \bar{\nu}$

\item \emph{$\beta^+$-decay}: $^1_1 \text{p} \longrightarrow ^1_0 \text{n} + ^{\phantom{+}0}_{+1} e^+ + ^0_0{\nu}$, or $\text{u} \longrightarrow \text{d} + e^+ + \nu$

\item \emph{interaction responsible for $\beta^\pm$-decays}: weak nuclear force
	
\end{compactitem}

\end{enumerate}
}